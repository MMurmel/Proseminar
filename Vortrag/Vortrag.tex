\documentclass{beamer}

%META-INFORMATION
\author{Maximilian Moeller}
\title{Maximaler Fluss in Flussnetzwerken}
\date{04.09.2020}

%PACKAGES
\usepackage[utf8]{inputenc}
\usepackage[german]{babel}
\usepackage[T1]{fontenc}
\usepackage{mathtools}
\usepackage{amsmath}
\usepackage{amsfonts}
\usepackage{amssymb}

%OPTIONS
%\setbeameroption{show notes}
\setbeamertemplate{footline}[frame number]
\beamertemplatenavigationsymbolsempty
\setbeamertemplate{section in toc}{\inserttocsectionnumber.~\inserttocsection}

%GRAPHING
\usepackage{tikz-network}
\usepackage{adjustbox}
\usepackage{subcaption}

%BIBTEX
\usepackage[numbers,square]{natbib}
\usepackage{url}
%smaller URLs
\renewcommand{\UrlFont}{\small\tt}

%ALIAS
\newcommand{\ff}{Ford-Fulkerson}
\newcommand{\pr}{Push/Relabel}

\begin{document}
\begin{frame}
\maketitle
\center\large Proseminar Theoretische Informatik 2020
\end{frame}

\begin{frame}
\frametitle{Gliederung}
\tableofcontents
\note[item]{Beweise nicht ausgeführt, nur Beweisideen}
alle Codesnippets und Definitionen aus \citep{Cormen09}
\end{frame}

%TODO: Motivation neu schreiben
\section*{Motivation}
\begin{frame}[plain]
\begin{center}
\begin{LARGE}
Motivation
\end{LARGE}
\end{center}
\end{frame}

\begin{frame}
\frametitle{Motivation}
\begin{columns}
\begin{column}{.4\textwidth}
\begin{itemize}
\item (Stoff-) Mengen auf mehreren Pfaden gleichzeitig transportiert
\item Pfade durch Kapazitäten beschränkt\pause
\item Beispiel Rechnernetze: Maximaler Durchsatz von A nach B?
\end{itemize}
\note[item]{!= Kapazitätsproblem}
\end{column}
\begin{column}{.6\textwidth}
\begin{tikzpicture}
\Vertex[label=$A$]{s}
\Vertex[label=$S1$, x=2, y=1]{1}
\Vertex[label=$S2$, x=2, y=-1]{2}
\Vertex[label=$S3$, x=4, y=1]{3}
\Vertex[label=$S4$, x=4, y=-1]{4}
\Vertex[label=$B$, x=6, y=0]{t}
\Edge[Direct, label=4](s)(1)
\Edge[Direct, label=7](s)(2)
\Edge[Direct, label=2](1)(3)
\Edge[Direct, label=4](2)(3)
\Edge[Direct, label=3](2)(4)
\Edge[Direct, label=6](3)(t)
\Edge[Direct, label=2](4)(t)
\end{tikzpicture}
\note[item]{Kapazitäten = Bandbreite d. Kanals}
\note[item]{maximal 8 Pakete}
\end{column}
\end{columns}
\end{frame}

\section{Maximaler-Fluss-Problem}
\begin{frame}[plain]
\begin{center}
\begin{LARGE}
Maximaler-Fluss-Problem
\end{LARGE}
\end{center}
\end{frame}

\subsection{Flussnetzwerke}
\begin{frame}
\frametitle{Flussnetzwerke}
\begin{columns}
\begin{column}{.45\textwidth}
\begin{itemize}
\item<1-> gerichteter Graph $G=(V,E)$
\item<2-> Kapazitäten $c(u,v) \geq 0$ für $u,v \in V$ (meist $\mathbb{N}$)
\item<3-> Quelle $s$ und Senke $t$
\item<4-> jeder Knoten $v$ liegt auf einem Pfad von $s$ nach $t$
\item<5-> Keine reflexiven Kanten (gleicher Start- und Zielknoten)
\end{itemize}
\note[item]{reflexive Kanten verboten: Beispiel Spannung beim Stromfluss}
\end{column}
\begin{column}{.6\textwidth}
\begin{adjustbox}{width=\textwidth}
\begin{tikzpicture}
\Vertex[label=$s$]{s}
\Vertex[label=$v_1$, x=2, y=1]{1}
\Vertex[label=$v_2$, x=2, y=-1]{2}
\Vertex[label=$v_3$, x=4, y=1]{3}
\Vertex[label=$v_4$, x=4, y=-1]{4}
\Vertex[label=$t$, x=6, y=0]{t}
\Edge[Direct, label=4](s)(1)
\Edge[Direct, label=7](s)(2)
\Edge[Direct, label=2](1)(3)
\Edge[Direct, label=4](2)(3)
\Edge[Direct, label=3](2)(4)
\Edge[Direct, label=6](3)(t)
\Edge[Direct, label=2](4)(t)
\end{tikzpicture}
\end{adjustbox}
\end{column}
\end{columns}
\end{frame}

\subsection{Fluss}
\begin{frame}
\frametitle{Fluss in Flussnetzwerken}
Fluss $f: V \times V \to \mathbb{N}$ erfüllt drei Bedingungen:
\vfill
\onslide<1->{\begin{equation}
\forall u,v\in V\colon (u,v)\notin E\Rightarrow f(u,v) = 0
\end{equation}}
\note[item]{Nur wo Rohre liegen kann Wasser fließen}
\onslide<2->{\begin{equation}
\forall u,v\in V\colon 0\leq f(u,v)\leq c(u,v)
\end{equation}}
\note[item]{Fluss hält Kapazitäten ein}
\onslide<3->{\begin{equation}
\forall u\in V\setminus\{s,t\}\colon\sum_{v\in V} f(v,u) = \sum_{v\in V} f(u,v)
\end{equation}}
\note[item]{Flusserhaltung, entspricht Kirchhoffscher Knotenregel}
\end{frame}

\begin{frame}
\frametitle{Notation/Beispiele}
$f(u,v)/c(u,v)$
\vfill
\note[item]{bei $f(u,v) = 0$ wird nur $c(u,v)$ angegeben}
\begin{center}
\begin{adjustbox}{width=0.8\textwidth}
\begin{tikzpicture}
\onslide<1->{
\Vertex[label=$s$]{s}
\Vertex[label=$v_1$, x=2, y=1]{1}
\Vertex[label=$v_2$, x=2, y=-1]{2}
\Vertex[label=$v_3$, x=4, y=1]{3}
\Vertex[label=$v_4$, x=4, y=-1]{4}
\Vertex[label=$t$, x=6, y=0]{t}}
\only<1>{
\Edge[Direct, label=1/4](s)(1)
\Edge[Direct, label=2/7](s)(2)
\Edge[Direct, label=1/2](1)(3)
\Edge[Direct, label=4](2)(3)
\Edge[Direct, label=2/3](2)(4)
\Edge[Direct, label=1/6](3)(t)
\Edge[Direct, label=2/2](4)(t)}
\only<2>{
\Edge[Direct, label=5/7](s)(1)
\Edge[Direct, label=1/1](1)(2)
\Edge[Direct, label=2/3](1)(3)
\Edge[Direct, label=2/100](1)(4)
\Edge[Direct, label=1/1](2)(4)
\Edge[Direct, label=2/2](3)(t)
\Edge[Direct, label=3/3](4)(t)}
\end{tikzpicture}
\note[item]{Auch anderer maximaler Fluss denkbar}
\end{adjustbox}
\end{center}
\end{frame}

\begin{frame}
\frametitle{Wert eines Flusses}
Wert eines Flusses $f$:
\begin{equation*}
\lvert f \rvert := \sum_{u\in V} f(s,u) - \sum_{u\in V} f(u,s)
\end{equation*}
\note[item]{Meist nur linker Term.}
\note[item]{"Netto"}
\note[item]{wegen Flusserhaltung gleich zu eingehendem Fluss in Senke}
\pause
\begin{figure}
\begin{adjustbox}{width=0.8\textwidth}
\begin{tikzpicture}
\Vertex[label=$s$]{s}
\Vertex[label=$u_1$, x=2, y=1]{1}
\Vertex[label=$u_2$, x=2, y=-1]{2}
\Vertex[label=$u_3$, x=4, y=1]{3}
\Vertex[label=$u_4$, x=4, y=-1]{4}
\Vertex[label=$t$, x=6, y=0]{t}
\Edge[Direct, label=1/4](s)(1)
\Edge[Direct, label=6/7](s)(2)
\Edge[Direct, label=1/2](1)(3)
\Edge[Direct, label=4/4](2)(3)
\Edge[Direct, label=2/3](2)(4)
\Edge[Direct, label=5/6](3)(t)
\Edge[Direct, label=2/2](4)(t)
\end{tikzpicture}
\end{adjustbox}
\caption{Ein Fluss $f$ mit $\lvert f\rvert = 7$}
\end{figure}
\end{frame}

\begin{frame}
\frametitle{Maximaler-Fluss-Problem}
\begin{itemize}
\item<1-> Gegeben: ein Flussnetzwerk $G = (V,E)$ und dessen Kapazitätsfunktion $c$.
\vfill
\item<2-> Gesucht: ein Fluss $f$, dessen Wert $\lvert f \rvert$ maximal für dieses Flussnetzwerk ist.
\end{itemize}
\end{frame}

\section{\ff -Algorithmen}
\begin{frame}[plain]
\begin{center}
\begin{LARGE}
\ff -Algorithmen
\end{LARGE}
\end{center}
\end{frame}

\subsection{Restnetzwerke}
\begin{frame}
\frametitle{Restnetzwerke}
Restkapazität:
\begin{equation*}
c_{f}(u,v) = 
\begin{cases}
c(u,v) - f(u,v) & \text{falls $(u,v)\in E$}\\
f(v,u) & \text{falls $(v,u) \in E$}\\
0 & \text{sonst}
\end{cases}
\end{equation*}
\pause
\vfill
Restkanten:
\begin{equation*}
E_f = \{ (u,v) \in V \times V \mid c_{f}(u,v) > 0\}
\end{equation*}
\vfill
\pause
Ein Fluss $f$ in einem Flussnetzwerk $G=(V,E)$ induziert das Restnetzwerk $G_{f}=(V,E_{f})$.
\note[item]{antiparallele Kanten möglich -> kein Flussnetzwerk}
\note[item]{Überleitung Beispiele}
\end{frame}

\begin{frame}
\frametitle{Restnetzwerke}
\begin{figure}[T]
\centering
\onslide<1->{
\begin{subfigure}{0.45\textwidth}
\begin{adjustbox}{width =\linewidth}
\begin{tikzpicture}
\Vertex[label=$s$]{s}
\Vertex[label=$v_1$, x=2, y=2]{1}
\Vertex[label=$v_2$, x=2, y=-2]{2}
\Vertex[label=$t$, x=4, y=0]{t}
\Edge[Direct, label=3/3](s)(1)
\Edge[Direct, label=2/3](s)(2)
\Edge[Direct, label=1](2)(1)
\Edge[Direct, label=3/7](1)(t)
\Edge[Direct, label=2/2](2)(t)
\end{tikzpicture}
\end{adjustbox}
\caption{Fluss $f$ in einem Flussnetzwerk $G = (V,E)$}
\end{subfigure}}
\hfill
\onslide<2->{
\begin{subfigure}{0.45\textwidth}
\begin{adjustbox}{width =\linewidth}
\begin{tikzpicture}
\Vertex[label=$s$]{s}
\Vertex[label=$v_1$, x=2, y=2]{1}
\Vertex[label=$v_2$, x=2, y=-2]{2}
\Vertex[label=$t$, x=4, y=0]{t}
\Edge[Direct, label=3](1)(s)
\Edge[Direct, label=1, bend=15](s)(2)
\Edge[Direct, label=2, bend=15](2)(s)
\Edge[Direct, label=1](2)(1)
\Edge[Direct, label=4, bend=15](1)(t)
\Edge[Direct, label=3, bend=15](t)(1)
\Edge[Direct, label=2](t)(2)
\end{tikzpicture}
\end{adjustbox}
\caption{durch $f$ induziertes Restnetzwerk $G_f = (V,E_f)$}
\end{subfigure}}
\end{figure}
\note[item]{Kanten in ursprünglicher Flussrichtung: wie viel mehr Fluss noch zulässig}
\note[item]{Kanten entgegen Flussrichtung: wie viel Fluss kann zurückgeschickt werden}
\note[item]{Kanten mit Restkapazität 0 nicht eingezeichnet}
Definition eines Flusses $f'$ im Restnetzwerk analog zu Fluss in Flussnetzwerken.
\end{frame}

\begin{frame}
\frametitle{Erhöhung eines Flusses}
Gegeben:
\begin{itemize}
\item Fluss $f$ im Flussnetzwerk $G=(V,E)$
\item Fluss $f'$ im Restnetzwerk $G_f=(V,E_f)$
\end{itemize}
\vfill
\onslide<2->{
Erhöhung von $f$ um $f'$:
\begin{equation*}
(f \uparrow f')(u,v)=
\begin{cases}
f(u,v) + f'(u,v) - f'(v,u) & \text{falls $(u,v) \in E$}\\
0 & \text{sonst}
\end{cases}
\end{equation*}}
\onslide<3->{
Wert der Erhöhung:
\begin{equation*}
\alert{\lvert f\uparrow f' \rvert = \lvert f \rvert + \lvert f'\rvert}
\end{equation*}}
\note[item]{Erhöhung erklären}
\onslide<4->{Beweis: über Umformen der Summen. $\square$}
\note[item]{Jetzt noch zu tun: finden eines Flusses in Gf -> Erweiterungspfade}
\end{frame}

\begin{frame}
\frametitle{Erhöhung eines Flusses}
\begin{figure}[T]
\centering
\begin{subfigure}{0.45\textwidth}
\begin{adjustbox}{width =\linewidth}
\begin{tikzpicture}
\onslide<1->{
\Vertex[label=$s$]{s}
\Vertex[label=$v_1$, x=2, y=2]{1}
\Vertex[label=$v_2$, x=2, y=-2]{2}
\Vertex[label=$t$, x=4, y=0]{t}}
\only<-2>{
\Edge[Direct, label=3/3](s)(1)
\Edge[Direct, label=2/3](s)(2)
\Edge[Direct, label=1](2)(1)
\Edge[Direct, label=3/7](1)(t)
\Edge[Direct, label=2/2](2)(t)}
\only<3>{
\Edge[Direct, label=3/3](s)(1)
\Edge[Direct, label=3/3](s)(2)
\Edge[Direct, label=1/1](2)(1)
\Edge[Direct, label=4/7](1)(t)
\Edge[Direct, label=2/2](2)(t)}
\end{tikzpicture}
\end{adjustbox}
\caption{\only<-2>{$f$ in $G$}\only<3>{$f\uparrow f'$ in $G$}}
\end{subfigure}
\hfill
\begin{subfigure}{0.45\textwidth}
\begin{adjustbox}{width =\linewidth}
\begin{tikzpicture}
\onslide<1->{
\Vertex[label=$s$]{s}
\Vertex[label=$v_1$, x=2, y=2]{1}
\Vertex[label=$v_2$, x=2, y=-2]{2}
\Vertex[label=$t$, x=4, y=0]{t}}
\only<1>{
\Edge[Direct, label=3](1)(s)
\Edge[Direct, label=1, bend=15](s)(2)
\Edge[Direct, label=2, bend=15](2)(s)
\Edge[Direct, label=1](2)(1)
\Edge[Direct, label=4, bend=15](1)(t)
\Edge[Direct, label=3, bend=15](t)(1)
\Edge[Direct, label=2](t)(2)}
\only<2>{
\Edge[Direct, label=3](1)(s)
\Edge[Direct, label=1/1, bend=15](s)(2)
\Edge[Direct, label=2, bend=15](2)(s)
\Edge[Direct, label=1/1](2)(1)
\Edge[Direct, label=1/4, bend=15](1)(t)
\Edge[Direct, label=3, bend=15](t)(1)
\Edge[Direct, label=2](t)(2)}
\only<3>{
\Edge[Direct, label=3](1)(s)
\Edge[Direct, label=3](2)(s)
\Edge[Direct, label=1](1)(2)
\Edge[Direct, label=3, bend=15](1)(t)
\Edge[Direct, label=4, bend=15](t)(1)
\Edge[Direct, label=2](t)(2)}
\end{tikzpicture}
\end{adjustbox}
\caption{\only<1>{$G_f$}\only<2>{$f'$ in $G_f$}\only<3>{$G_{f\uparrow f'}$}}
\end{subfigure}
\end{figure}
\end{frame}

\subsection{Erweiterungspfade}
\begin{frame}
\frametitle{Erweiterungspfade}
Gegeben:
\begin{itemize}
\item Pfad $p$ von $s$ nach $t$ im Restnetzwerk $G_f$
\end{itemize}
\vfill
\onslide<2->{
Pfadkapazität:
\begin{equation*}
c_f(p) = min\{ c_f(u,v)\mid (u,v) \text{ liegt auf $p$}\}
\end{equation*}}
\onslide<3->{
Definiere Fluss $f_p$ entlang von $p$:
\begin{equation*}
f_p(u,v) = \begin{cases}c_f(p) & \text{falls $(u,v)$ auf $p$ liegt}\\0 & \text{sonst}\end{cases}
\end{equation*}}
\onslide<4->{
Damit:
\begin{equation*}
\lvert f\uparrow f_p \rvert = \lvert f \rvert + \lvert f_p\rvert > \lvert f\rvert
\end{equation*}}
\note[item]{alles zusammen für \ff-Algorithmus}
\end{frame}

\subsection{generischer Algorithmus}
\begin{frame}
\frametitle{\ff-Basisalgorithmus}
\begin{adjustbox}{width = \textwidth}
\includegraphics{../Grafiken/FF-Basis.png}
\end{adjustbox}
\note[item]{Jack Edmonds und Richard Karp, 1970}
\end{frame}

\begin{frame}
\frametitle{\ff-Analyse}
Eigenschaften des generischen \ff-Algorithmus:
\begin{itemize}
\item<1-> Terminiert stets bei Kantengewichten aus $\mathbb{N}$ (und $\mathbb{Q}$)
\item<2-> Fluss ist maximal wegen \alert{maxflow-mincut-Theorem} 
\begin{itemize}
\item u.A.: $\lvert f\rvert$ ist maximal $\Leftrightarrow$ kein Erweiterungspfad in $G_{f}$
\end{itemize}
\item<3-> Laufzeit: $\mathcal{O}(\lvert f_{max}\rvert \cdot E)$ für einen maximalen Fluss $f_{max}$
\item<4-> Problem: schlechte Wegewahl möglich
\end{itemize}
\note[item]{bei rationalen Zahlen Multiplikation mit Hauptnenner}
\note[item]{Terminierung bzw. Konvergenz bei reellen Zahlen nicht sichergestellt}
\end{frame}

\begin{frame}
\frametitle{schlechte Wegewahl}
\vfill
\begin{figure}[T]
\centering
\begin{subfigure}{0.45\textwidth}
\begin{adjustbox}{width =\linewidth}
\begin{tikzpicture}
\onslide<1->{
\Vertex[label=$s$]{s}
\Vertex[label=$v_1$, x=2, y=1]{1}
\Vertex[label=$v_2$, x=2, y=-1]{2}
\Vertex[label=$t$, x=4]{t}}
\only<1-2>{
\Edge[Direct, label=100](s)(1)
\Edge[Direct, label=100](s)(2)
\Edge[Direct, label=1](1)(2)
\Edge[Direct, label=100](1)(t)
\Edge[Direct, label=100](2)(t)}
\only<3-4>{
\Edge[Direct, label=1/100](s)(1)
\Edge[Direct, label=100](s)(2)
\Edge[Direct, label=1/1](1)(2)
\Edge[Direct, label=100](1)(t)
\Edge[Direct, label=1/100](2)(t)}
\only<5-6>{
\Edge[Direct, label=1/100](s)(1)
\Edge[Direct, label=1/100](s)(2)
\Edge[Direct, label=1/1](1)(2)
\Edge[Direct, label=1/100](1)(t)
\Edge[Direct, label=1/100](2)(t)}
\end{tikzpicture}
\end{adjustbox}
\caption{$G$ mit $f$}
\end{subfigure}
\hfill
\begin{subfigure}{0.45\textwidth}
\begin{adjustbox}{width =\linewidth}
\begin{tikzpicture}
\onslide<1->{
\Vertex[label=$s$]{s}
\Vertex[label=$v_1$, x=2, y=1]{1}
\Vertex[label=$v_2$, x=2, y=-1]{2}
\Vertex[label=$t$, x=4]{t}}
\only<1>{
\Edge[Direct, label=100](s)(1)
\Edge[Direct, label=100](s)(2)
\Edge[Direct, label=1](1)(2)
\Edge[Direct, label=100](1)(t)
\Edge[Direct, label=100](2)(t)}
\only<2>{
\Edge[Direct, label=100, color=red](s)(1)
\Edge[Direct, label=100](s)(2)
\Edge[Direct, label=1, color=red](1)(2)
\Edge[Direct, label=100](1)(t)
\Edge[Direct, label=100, color=red](2)(t)}
\only<3>{
\Edge[Direct, label=99, bend=15](s)(1)
\Edge[Direct, label=1, bend=15](1)(s)
\Edge[Direct, label=100](s)(2)
\Edge[Direct, label=1](2)(1)
\Edge[Direct, label=100](1)(t)
\Edge[Direct, label=99, bend=15](2)(t)
\Edge[Direct, label=1, bend=15](t)(2)}
\only<4>{
\Edge[Direct, label=99, bend=15](s)(1)
\Edge[Direct, label=1, bend=15](1)(s)
\Edge[Direct, label=100, color=red](s)(2)
\Edge[Direct, label=1, color=red](2)(1)
\Edge[Direct, label=100, color=red](1)(t)
\Edge[Direct, label=99, bend=15](2)(t)
\Edge[Direct, label=1, bend=15](t)(2)}
\only<5>{
\Edge[Direct, label=99, bend=15](s)(1)
\Edge[Direct, label=1, bend=15](1)(s)
\Edge[Direct, label=99, bend=15](s)(2)
\Edge[Direct, label=1, bend=15](2)(s)
\Edge[Direct, label=1](1)(2)
\Edge[Direct, label=99, bend=15](1)(t)
\Edge[Direct, label=1, bend=15](t)(1)
\Edge[Direct, label=99, bend=15](2)(t)
\Edge[Direct, label=1, bend=15](t)(2)}
\only<6>{
\Edge[Direct, label=99, bend=15, color=red](s)(1)
\Edge[Direct, label=1, bend=15](1)(s)
\Edge[Direct, label=99, bend=15](s)(2)
\Edge[Direct, label=1, bend=15](2)(s)
\Edge[Direct, label=1, color=red](1)(2)
\Edge[Direct, label=99, bend=15](1)(t)
\Edge[Direct, label=1, bend=15](t)(1)
\Edge[Direct, label=99, bend=15, color=red](2)(t)
\Edge[Direct, label=1, bend=15](t)(2)}
\end{tikzpicture}
\end{adjustbox}
\caption{$G_f$}
\end{subfigure}
\caption{gewählter Erweiterungspfad in rot}
\end{figure}
\end{frame}

\begin{frame}
\frametitle{Edmonds-Karp-Algorithmus}
Wahl des Erweiterungspfades als ein kürzester Pfad (minimale Pfadlänge)
\begin{itemize}
\item $\mathcal{O}(V\cdot E^2)$
\end{itemize}
\note[item]{kürzester Pfad meint Anzahl an Kanten}
\note[item]{Breitensuche}
\end{frame}

\section{\pr -Algorithmen}
\begin{frame}[plain]
\begin{center}
\begin{LARGE}
\pr -Algorithmen
\end{LARGE}
\end{center}
\end{frame}

\subsection{Operationen}
\begin{frame}
\frametitle{\pr -Grundlagen}
\note[item]{lokaler als \ff}
\note[item]{spezielle Parameter der Knoten}
\onslide<1->{
arbeitet mit Vorfluss: 
\begin{equation*}
\forall u \in V\setminus\{ s\} :\sum_{v\in V} f(v,u) \geq \sum_{v\in V} f(u,v)
\end{equation*}}
\onslide<2->{
Flussüberschuss $e(u)$ des Knotens $u \in V$: 
\begin{equation*}
e(u) = \sum_{v\in V} f(v,u) - \sum_{v\in V} f(u,v)
\end{equation*}}
\note[item]{Knoten ist überflutet bei $e(u) > 0$}
\onslide<3->{
Höhenfunktion $h: V\to \mathbb{N}$:
\begin{equation*}
\forall (u,v) \in E_f:h(u) \leq h(v) + 1
\end{equation*}}
\onslide<4->{$h(S) = \lvert V\rvert$, $h(t)=0$}
\end{frame}

\begin{frame}
\frametitle{Push-Operation}
\begin{adjustbox}{width = \textwidth}
\includegraphics{../Grafiken/Push.png}
\end{adjustbox}
\end{frame}

\begin{frame}
\frametitle{Relabel-Operation}
\begin{adjustbox}{width = \textwidth}
\includegraphics{../Grafiken/Relabel.png}
\end{adjustbox}
\end{frame}

\subsection{generischer Algorithmus}
\begin{frame}
\frametitle{generischer \pr -Algorithmus}
\centering
\begin{adjustbox}{width = 0.5\textwidth}
\includegraphics{../Grafiken/Init-Preflow.png}
\end{adjustbox}
\begin{adjustbox}{width = \textwidth}
\includegraphics{../Grafiken/Generic-PR.png}
\end{adjustbox}
\end{frame}

\begin{frame}
\frametitle{\pr -Beispiel}
\begin{figure}
\centering
\begin{subfigure}{0.45\textwidth}
\begin{adjustbox}{width =\linewidth}
\begin{tikzpicture}
\only<1>{
\Vertex[label=$s$]{s}
\Vertex[label=$v_1$, x=2, y=1]{1}
\Vertex[label=$v_2$, x=2, y=-1]{2}
\Vertex[label=$t$, x=4]{t}
\Edge[Direct, label=3](s)(1)
\Edge[Direct, label=4](s)(2)
\Edge[Direct, label=3](1)(2)
\Edge[Direct, label=1](1)(t)
\Edge[Direct, label=5](2)(t)}
\only<2>{
\Vertex[label=$s$]{s}
\Vertex[label=$v_1$, x=2, y=1]{1}
\Vertex[label=$v_2$, x=2, y=-1]{2}
\Vertex[label=$t$, x=4]{t}
\Edge[Direct, label=3/3](s)(1)
\Edge[Direct, label=4/4](s)(2)
\Edge[Direct, label=3](1)(2)
\Edge[Direct, label=1](1)(t)
\Edge[Direct, label=5](2)(t)}
\only<3>{
\Vertex[label=$s$]{s}
\Vertex[label=$v_1$, x=2, y=1, color=lime]{1}
\Vertex[label=$v_2$, x=2, y=-1]{2}
\Vertex[label=$t$, x=4]{t}
\Edge[Direct, label=3/3](s)(1)
\Edge[Direct, label=4/4](s)(2)
\Edge[Direct, label=3](1)(2)
\Edge[Direct, label=1](1)(t)
\Edge[Direct, label=5](2)(t)}
\only<4>{
\Vertex[label=$s$]{s}
\Vertex[label=$v_1$, x=2, y=1, color=lime]{1}
\Vertex[label=$v_2$, x=2, y=-1]{2}
\Vertex[label=$t$, x=4]{t}
\Edge[Direct, label=3/3](s)(1)
\Edge[Direct, label=4/4](s)(2)
\Edge[Direct, label=3, color=lime](1)(2)
\Edge[Direct, label=1, color=lime](1)(t)
\Edge[Direct, label=5](2)(t)}
\only<5>{
\Vertex[label=$s$]{s}
\Vertex[label=$v_1$, x=2, y=1, color=lime]{1}
\Vertex[label=$v_2$, x=2, y=-1]{2}
\Vertex[label=$t$, x=4]{t}
\Edge[Direct, label=3/3](s)(1)
\Edge[Direct, label=4/4](s)(2)
\Edge[Direct, label=3, color=lime](1)(2)
\Edge[Direct, label=1/1](1)(t)
\Edge[Direct, label=5](2)(t)}
\only<6>{
\Vertex[label=$s$]{s}
\Vertex[label=$v_1$, x=2, y=1]{1}
\Vertex[label=$v_2$, x=2, y=-1, color=lime]{2}
\Vertex[label=$t$, x=4]{t}
\Edge[Direct, label=3/3](s)(1)
\Edge[Direct, label=4/4](s)(2)
\Edge[Direct, label=2/3](1)(2)
\Edge[Direct, label=1/1](1)(t)
\Edge[Direct, label=5](2)(t)}
\only<7>{
\Vertex[label=$s$]{s}
\Vertex[label=$v_1$, x=2, y=1]{1}
\Vertex[label=$v_2$, x=2, y=-1, color=lime]{2}
\Vertex[label=$t$, x=4]{t}
\Edge[Direct, label=3/3](s)(1)
\Edge[Direct, label=4/4](s)(2)
\Edge[Direct, label=2/3](1)(2)
\Edge[Direct, label=1/1](1)(t)
\Edge[Direct, label=5, color=lime](2)(t)}
\only<8>{
\Vertex[label=$s$]{s}
\Vertex[label=$v_1$, x=2, y=1]{1}
\Vertex[label=$v_2$, x=2, y=-1, color=lime]{2}
\Vertex[label=$t$, x=4]{t}
\Edge[Direct, label=3/3](s)(1)
\Edge[Direct, label=4/4](s)(2)
\Edge[Direct, label=2/3](1)(2)
\Edge[Direct, label=1/1](1)(t)
\Edge[Direct, label=5/5](2)(t)}
\only<9>{
\Vertex[label=$s$]{s}
\Vertex[label=$v_1$, x=2, y=1]{1}
\Vertex[label=$v_2$, x=2, y=-1, color=lime]{2}
\Vertex[label=$t$, x=4]{t}
\Edge[Direct, label=3/3](s)(1)
\Edge[Direct, label=4/4](s)(2)
\Edge[Direct, label=2/3, color=lime](1)(2)
\Edge[Direct, label=1/1](1)(t)
\Edge[Direct, label=5/5](2)(t)}
\only<10>{
\Vertex[label=$s$]{s}
\Vertex[label=$v_1$, x=2, y=1, color=lime]{1}
\Vertex[label=$v_2$, x=2, y=-1]{2}
\Vertex[label=$t$, x=4]{t}
\Edge[Direct, label=3/3](s)(1)
\Edge[Direct, label=4/4](s)(2)
\Edge[Direct, label=1/3](1)(2)
\Edge[Direct, label=1/1](1)(t)
\Edge[Direct, label=5/5](2)(t)}
\only<11>{
\Vertex[label=$s$]{s}
\Vertex[label=$v_1$, x=2, y=1]{1}
\Vertex[label=$v_2$, x=2, y=-1, color=lime]{2}
\Vertex[label=$t$, x=4]{t}
\Edge[Direct, label=3/3](s)(1)
\Edge[Direct, label=4/4](s)(2)
\Edge[Direct, label=2/3, color=lime](1)(2)
\Edge[Direct, label=1/1](1)(t)
\Edge[Direct, label=5/5](2)(t)}
\only<12>{
\Vertex[label=$s$]{s}
\Vertex[label=$v_1$, x=2, y=1, color=lime]{1}
\Vertex[label=$v_2$, x=2, y=-1]{2}
\Vertex[label=$t$, x=4]{t}
\Edge[Direct, label=3/3](s)(1)
\Edge[Direct, label=4/4](s)(2)
\Edge[Direct, label=1/3](1)(2)
\Edge[Direct, label=1/1](1)(t)
\Edge[Direct, label=5/5](2)(t)}
\only<13>{
\Vertex[label=$s$]{s}
\Vertex[label=$v_1$, x=2, y=1, color=lime]{1}
\Vertex[label=$v_2$, x=2, y=-1]{2}
\Vertex[label=$t$, x=4]{t}
\Edge[Direct, label=3/3, color=lime](s)(1)
\Edge[Direct, label=4/4](s)(2)
\Edge[Direct, label=1/3](1)(2)
\Edge[Direct, label=1/1](1)(t)
\Edge[Direct, label=5/5](2)(t)}
\only<14>{
\Vertex[label=$s$]{s}
\Vertex[label=$v_1$, x=2, y=1]{1}
\Vertex[label=$v_2$, x=2, y=-1]{2}
\Vertex[label=$t$, x=4]{t}
\Edge[Direct, label=2/3](s)(1)
\Edge[Direct, label=4/4](s)(2)
\Edge[Direct, label=1/3](1)(2)
\Edge[Direct, label=1/1](1)(t)
\Edge[Direct, label=5/5](2)(t)}
\end{tikzpicture}
\end{adjustbox}
\end{subfigure}
\begin{subfigure}{0.45\textwidth}
\begin{adjustbox}{width =\linewidth}
\begin{tikzpicture}
\onslide<1->{
\draw[step=1cm,gray,very thin] (0,0) grid (5,5);
\draw (5,0) node[anchor=north west] {$u \in V$};
\draw[thick,->] (0,0) -- (0,5) node[anchor=south east] {$h(u)$};
\foreach \y in {0,1,2,3,4}
    \draw (1pt,\y cm) -- (-1pt,\y cm) node[anchor=east] {$\y$};}
\only<1>{
\Vertex[label=$s$, x=1, y=0]{s}
\Vertex[label=$v_1$, x=2, y=0]{1}
\Vertex[label=$v_2$, x=3, y=0]{2}
\Vertex[label=$t$, x=4, y=0]{t}
\draw (5,5.5) node[anchor=south west, text=red] {\large u.e};
\draw (1,5.5) node[anchor=south, text=red] {\large 0};
\draw (2,5.5) node[anchor=south, text=red] {\large 0};
\draw (3,5.5) node[anchor=south, text=red] {\large 0};
\draw (4,5.5) node[anchor=south, text=red] {\large 0};}
\only<2>{
\Vertex[label=$s$, x=1, y=4]{s}
\Vertex[label=$v_1$, x=2, y=0]{1}
\Vertex[label=$v_2$, x=3, y=0]{2}
\Vertex[label=$t$, x=4, y=0]{t}
\draw (5,5.5) node[anchor=south west, text=red] {\large u.e};
\draw (1,5.5) node[anchor=south, text=red] {\large -7};
\draw (2,5.5) node[anchor=south, text=red] {\large 3};
\draw (3,5.5) node[anchor=south, text=red] {\large 4};
\draw (4,5.5) node[anchor=south, text=red] {\large 0};}
\only<3>{
\Vertex[label=$s$, x=1, y=4]{s}
\Vertex[label=$v_1$, x=2, y=0, color=lime]{1}
\Vertex[label=$v_2$, x=3, y=0]{2}
\Vertex[label=$t$, x=4, y=0]{t}
\draw (5,5.5) node[anchor=south west, text=red] {\large u.e};
\draw (1,5.5) node[anchor=south, text=red] {\large -7};
\draw (2,5.5) node[anchor=south, text=red] {\large 3};
\draw (3,5.5) node[anchor=south, text=red] {\large 4};
\draw (4,5.5) node[anchor=south, text=red] {\large 0};}
\only<4>{
\Vertex[label=$s$, x=1, y=4]{s}
\Vertex[label=$v_1$, x=2, y=1, color=lime]{1}
\Vertex[label=$v_2$, x=3, y=0]{2}
\Vertex[label=$t$, x=4, y=0]{t}
\draw (5,5.5) node[anchor=south west, text=red] {\large u.e};
\draw (1,5.5) node[anchor=south, text=red] {\large -7};
\draw (2,5.5) node[anchor=south, text=red] {\large 3};
\draw (3,5.5) node[anchor=south, text=red] {\large 4};
\draw (4,5.5) node[anchor=south, text=red] {\large 0};}
\only<5>{
\Vertex[label=$s$, x=1, y=4]{s}
\Vertex[label=$v_1$, x=2, y=1, color=lime]{1}
\Vertex[label=$v_2$, x=3, y=0]{2}
\Vertex[label=$t$, x=4, y=0]{t}
\draw (5,5.5) node[anchor=south west, text=red] {\large u.e};
\draw (1,5.5) node[anchor=south, text=red] {\large -7};
\draw (2,5.5) node[anchor=south, text=red] {\large 2};
\draw (3,5.5) node[anchor=south, text=red] {\large 4};
\draw (4,5.5) node[anchor=south, text=red] {\large 1};}
\only<6>{
\Vertex[label=$s$, x=1, y=4]{s}
\Vertex[label=$v_1$, x=2, y=1]{1}
\Vertex[label=$v_2$, x=3, y=0, color=lime]{2}
\Vertex[label=$t$, x=4, y=0]{t}
\draw (5,5.5) node[anchor=south west, text=red] {\large u.e};
\draw (1,5.5) node[anchor=south, text=red] {\large -7};
\draw (2,5.5) node[anchor=south, text=red] {\large 0};
\draw (3,5.5) node[anchor=south, text=red] {\large 6};
\draw (4,5.5) node[anchor=south, text=red] {\large 1};}
\only<7>{
\Vertex[label=$s$, x=1, y=4]{s}
\Vertex[label=$v_1$, x=2, y=1]{1}
\Vertex[label=$v_2$, x=3, y=1, color=lime]{2}
\Vertex[label=$t$, x=4, y=0]{t}
\draw (5,5.5) node[anchor=south west, text=red] {\large u.e};
\draw (1,5.5) node[anchor=south, text=red] {\large -7};
\draw (2,5.5) node[anchor=south, text=red] {\large 0};
\draw (3,5.5) node[anchor=south, text=red] {\large 6};
\draw (4,5.5) node[anchor=south, text=red] {\large 1};}
\only<8>{
\Vertex[label=$s$, x=1, y=4]{s}
\Vertex[label=$v_1$, x=2, y=1]{1}
\Vertex[label=$v_2$, x=3, y=1, color=lime]{2}
\Vertex[label=$t$, x=4, y=0]{t}
\draw (5,5.5) node[anchor=south west, text=red] {\large u.e};
\draw (1,5.5) node[anchor=south, text=red] {\large -7};
\draw (2,5.5) node[anchor=south, text=red] {\large 0};
\draw (3,5.5) node[anchor=south, text=red] {\large 1};
\draw (4,5.5) node[anchor=south, text=red] {\large 6};}
\only<9>{
\Vertex[label=$s$, x=1, y=4]{s}
\Vertex[label=$v_1$, x=2, y=1]{1}
\Vertex[label=$v_2$, x=3, y=2, color=lime]{2}
\Vertex[label=$t$, x=4, y=0]{t}
\draw (5,5.5) node[anchor=south west, text=red] {\large u.e};
\draw (1,5.5) node[anchor=south, text=red] {\large -7};
\draw (2,5.5) node[anchor=south, text=red] {\large 0};
\draw (3,5.5) node[anchor=south, text=red] {\large 1};
\draw (4,5.5) node[anchor=south, text=red] {\large 6};}
\only<10>{
\Vertex[label=$s$, x=1, y=4]{s}
\Vertex[label=$v_1$, x=2, y=1, color=lime]{1}
\Vertex[label=$v_2$, x=3, y=2]{2}
\Vertex[label=$t$, x=4, y=0]{t}
\draw (5,5.5) node[anchor=south west, text=red] {\large u.e};
\draw (1,5.5) node[anchor=south, text=red] {\large -7};
\draw (2,5.5) node[anchor=south, text=red] {\large 1};
\draw (3,5.5) node[anchor=south, text=red] {\large 0};
\draw (4,5.5) node[anchor=south, text=red] {\large 6};}
\only<11>{
\Vertex[label=$s$, x=1, y=4]{s}
\Vertex[label=$v_1$, x=2, y=3]{1}
\Vertex[label=$v_2$, x=3, y=2, color=lime]{2}
\Vertex[label=$t$, x=4, y=0]{t}
\draw (5,5.5) node[anchor=south west, text=red] {\large u.e};
\draw (1,5.5) node[anchor=south, text=red] {\large -7};
\draw (2,5.5) node[anchor=south, text=red] {\large 0};
\draw (3,5.5) node[anchor=south, text=red] {\large 1};
\draw (4,5.5) node[anchor=south, text=red] {\large 6};}
\only<12>{
\Vertex[label=$s$, x=1, y=4]{s}
\Vertex[label=$v_1$, x=2, y=3, color=lime]{1}
\Vertex[label=$v_2$, x=3, y=4]{2}
\Vertex[label=$t$, x=4, y=0]{t}
\draw (5,5.5) node[anchor=south west, text=red] {\large u.e};
\draw (1,5.5) node[anchor=south, text=red] {\large -7};
\draw (2,5.5) node[anchor=south, text=red] {\large 1};
\draw (3,5.5) node[anchor=south, text=red] {\large 0};
\draw (4,5.5) node[anchor=south, text=red] {\large 6};}
\only<13>{
\Vertex[label=$s$, x=1, y=4]{s}
\Vertex[label=$v_1$, x=2, y=5, color=lime]{1}
\Vertex[label=$v_2$, x=3, y=4]{2}
\Vertex[label=$t$, x=4, y=0]{t}
\draw (5,5.5) node[anchor=south west, text=red] {\large u.e};
\draw (1,5.5) node[anchor=south, text=red] {\large -7};
\draw (2,5.5) node[anchor=south, text=red] {\large 1};
\draw (3,5.5) node[anchor=south, text=red] {\large 0};
\draw (4,5.5) node[anchor=south, text=red] {\large 6};}
\only<14    >{
\Vertex[label=$s$, x=1, y=4]{s}
\Vertex[label=$v_1$, x=2, y=5]{1}
\Vertex[label=$v_2$, x=3, y=4]{2}
\Vertex[label=$t$, x=4, y=0]{t}
\draw (5,5.5) node[anchor=south west, text=red] {\large u.e};
\draw (1,5.5) node[anchor=south, text=red] {\large -6};
\draw (2,5.5) node[anchor=south, text=red] {\large 0};
\draw (3,5.5) node[anchor=south, text=red] {\large 0};
\draw (4,5.5) node[anchor=south, text=red] {\large 6};}
\end{tikzpicture}
\end{adjustbox}
\end{subfigure}
\end{figure}
\note[item]{initialize preflow abgeschlossen}
\note[item]{in anderer Farbe der für die nächste Operation betrachtete Knoten (push nicht anwendbar, aber relabel -> auf t.h+1= v2.h+1 setzen)}
\end{frame}

\begin{frame}
\frametitle{\pr -Analyse}
Eigenschaften des generischen \pr-Algorithmus:
\begin{itemize}
\item<1-> Vorfluss $f$ ist bei Terminierung ein Fluss.
\note[item]{Beweis über mehrere Beobachtungen zur Höhenfunktion und eine Schleifeninvariante}
\note[item]{Terminierung über Abschätzung der benötigten Operationen}
\item<2-> insgesamt $\mathcal{O}(V^2 \cdot E)$
\item<3-> Problem: beliebige Reihenfolge ineffizient
\item<4-> Verbesserung: Relabel-to-Front-Algorithmus \begin{itemize}
    \item $\mathcal{O}(V^3)$
\end{itemize}
\note[item]{geschickte Wahl der Datenstruktur und der ausgeführten Operationen}
\end{itemize}
\end{frame}

%TODO: Gegenüberstellung (Vergleich) Algorithmen
%TODO: Zusammenfassung Vortrag

\begin{frame}
\frametitle{Literatur}
\nocite{PR-Visual}
\bibliography{../../references}
\bibliographystyle{alpha}
\end{frame}

\end{document}
