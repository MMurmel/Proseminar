\documentclass{beamer}

%\usepackage{pgfpages}
%\setbeameroption{show notes on second screen=right}

%DEFAULT
\usepackage[utf8]{inputenc}
\usepackage[german]{babel}
\usepackage[T1]{fontenc}
\usepackage{amsmath}
\usepackage{amsfonts}
\usepackage{amssymb}

%theming
\usetheme{Dresden}
\usecolortheme{dolphin}
%\setbeamertemplate{note page}{\pagecolor{yellow!5}\insertnote}

%GRAPHING
\usepackage{tikz-network}

%BIBTEX
\usepackage[numbers,square]{natbib}

%META-INFORMATION
\author{Maximilian Moeller}
\title{Maximaler Fluss in Flussnetzwerken}
\date{\today}

%ALIAS
\newcommand{\ff}{\textsc{Ford-Fulkerson}}
\newcommand{\pr}{Push/Relabel}

\begin{document}
\begin{frame}
\maketitle
\center\large Proseminar Theoretische Informatik 2020
\end{frame}

\begin{frame}
\frametitle{Gliederung}
\tableofcontents
\end{frame}


\section{Motivation}
\begin{frame}
\begin{columns}
\begin{column}{.4\textwidth}
\begin{itemize}
\item (Stoff-) Mengen auf mehreren Pfaden gleichzeitig transportiert
\item Pfade durch Kapazitäten beschränkt\pause
\item Beispiel Rechnernetze: Maximale Anzahl der gleichzeitig transportierten Pakete von A nach B?
\end{itemize}
\note{!= Kapazitätsproblem}
\note{Alle Definitionen etc. aus \citep{Co3rmen09}}
\end{column}

\begin{column}{.6\textwidth}
\begin{tikzpicture}
\Vertex[label=$A$]{s}
\Vertex[label=$S1$, x=2, y=1]{1}
\Vertex[label=$S2$, x=2, y=-1]{2}
\Vertex[label=$S3$, x=4, y=1]{3}
\Vertex[label=$S4$, x=4, y=-1]{4}
\Vertex[label=$t$, x=6, y=0]{t}

\Edge[Direct, label=4](s)(1)
\Edge[Direct, label=7](s)(2)
\Edge[Direct, label=2](1)(3)
\Edge[Direct, label=4](2)(3)
\Edge[Direct, label=3](2)(4)
\Edge[Direct, label=6](3)(t)
\Edge[Direct, label=2](4)(t)
\end{tikzpicture}
\note{Kapazitäten = Bandbreite d. Kanals}
\note{maximal 8 Pakete}
\end{column}
\end{columns}
\end{frame}


\section{Grundlagen}
\subsection{Flussnetzwerke}
\begin{frame}
\frametitle{Flussnetzwerke}
\textit{
\begin{itemize}
\item verbotene Spezialfälle (anti-parallel, reflexiv)
\end{itemize}
}
\end{frame}

\begin{frame}
\begin{columns}
\begin{column}{.5\textwidth}
\begin{itemize}
\item gerichteter Graph $G=(V,E)$
\item Kapazitäten $c: E \to \mathbb{N}$\\$\forall (u,v)\in E: c(u,v) \geq 0$
\item Quelle $s$ und Senke $t$
\item Pfade für alle Knoten\\$\forall v \in V:s \to v \to t$
\end{itemize}
\note{•}
\end{column}

\begin{column}{.6\textwidth}
\begin{tikzpicture}
\Vertex[label=$s$]{s}
\Vertex[label=$v_1$, x=2, y=1]{1}
\Vertex[label=$v_2$, x=2, y=-1]{2}
\Vertex[label=$v_3$, x=4, y=1]{3}
\Vertex[label=$v_4$, x=4, y=-1]{4}
\Vertex[label=$B$, x=6, y=0]{t}

\Edge[Direct, label=4](s)(1)
\Edge[Direct, label=7](s)(2)
\Edge[Direct, label=2](1)(3)
\Edge[Direct, label=4](2)(3)
\Edge[Direct, label=3](2)(4)
\Edge[Direct, label=6](3)(t)
\Edge[Direct, label=2](4)(t)
\end{tikzpicture}
\note{•}
\end{column}

\end{columns}
\end{frame}

\begin{frame}
\frametitle{Verbotene Spezialfälle}
\begin{itemize}
\item FN mit antiparallelen Kanten
\item FN mit mehreren Quellen/Senken
\end{itemize}
\end{frame}

\begin{frame}
\frametitle{verbotene Spezialfälle}
\begin{columns}
\begin{column}{.2\textwidth}
\begin{center}

\begin{tikzpicture}
\Vertex[label=$s$]{s}
\Vertex[label=$v_1$, x=2, y=1]{1}
\Vertex[label=$v_2$, x=2, y=-1]{2}
\Vertex[label=$v_3$, x=4, y=1]{3}
\Vertex[label=$v_4$, x=4, y=-1]{4}
\Vertex[label=$B$, x=6, y=0]{t}

\Edge[Direct, label=4](s)(1)
\Edge[Direct, label=7](s)(2)
\Edge[Direct, label=2](1)(3)
\Edge[Direct, label=4](2)(3)
\Edge[Direct, label=3](2)(4)
\Edge[Direct, label=6](3)(t)
\Edge[Direct, label=2](4)(t)
\end{tikzpicture}

\end{center}
\end{column}

\begin{column}{.48\textwidth}
\begin{tikzpicture}
\Vertex[label=$s$]{s}
\Vertex[label=$v_1$, x=2, y=1]{1}
\Vertex[label=$v_2$, x=2, y=-1]{2}
\Vertex[label=$v_3$, x=4, y=1]{3}
\Vertex[label=$v_4$, x=4, y=-1]{4}
\Vertex[label=$B$, x=6, y=0]{t}

\Edge[Direct, label=4](s)(1)
\Edge[Direct, label=7](s)(2)
\Edge[Direct, label=2](1)(3)
\Edge[Direct, label=4](2)(3)
\Edge[Direct, label=3](2)(4)
\Edge[Direct, label=6](3)(t)
\Edge[Direct, label=2](4)(t)
\end{tikzpicture}
\end{column}
\end{columns}
\end{frame}

\subsection{Fluss}
\begin{frame}
\frametitle{Fluss in Flussnetzwerken}
\textit{
\begin{itemize}
\item Grafik: Flussnetzwerk mit Fluss
\item Notation einführen
\item Anforderungen an einen Fluss
\end{itemize}
}
\end{frame}

\begin{frame}
\frametitle{Maximaler-Fluss-Problem}
\textit{
\begin{itemize}
\item Wert eines Flusses
\item Grafiken mit leicht erkennbaren Werten der Flüsse
\item maximaler-Fluss-Problem
\end{itemize}
}
\end{frame}


\section{weitere Theorie}
\subsection{Restnetzwerke}
\begin{frame}
\frametitle{Restnetzwerke}
\textit{
\begin{itemize}
\item Grafik: Restnetzwerk aus Flussnetzwerk und Fluss
\item Erklärung: RN sind keine FN, aber Fluss trotzdem analog
\item Erhöhung $f\uparrow f'$
\end{itemize}
}
\end{frame}

\subsection{Erweiterungspfade}
\begin{frame}
\frametitle{Erweiterungspfade}
\textit{
\begin{itemize}
\item Grafik: Erweiterungspfad in $G_{f}$
\item Erklärung: $f_{p}$ und $\lvert f\uparrow f_{p} \rvert > \lvert f\rvert$
\item Überleitung zu \ff
\end{itemize}
}
\end{frame}


\section{Algorithmen}
\subsection{Ford-Fulkerson}
\begin{frame}
\frametitle{\ff}
\textit{
\begin{itemize}
\item Erklärung Algorithmus mit Vorwissen aus RN und Erweiterungspfade
\item maximal? -> ja, wegen maxflow-mincut-Theorem (Erklärung reduzieren auf $\lvert f\rvert$ max. $\Leftrightarrow$ kein Pfad in $G_{f}$)
\item Terminierung
\item Verbesserungen (Edmonds-Karp-Algorithmus)
\end{itemize}
}
\end{frame}

\subsection{Push-Relabel}
\begin{frame}
\frametitle{\pr}
\textit{
\begin{itemize}
\item Intuition
\item Vorfluss -> verletzt ggf. Flusserhaltung
\item Höhe
\item Operationen
\item generischer \pr -Algorithmus
\item Verbesserungen (Relabel-to-Front)
\end{itemize}
}
\end{frame}

\subsection{Vergleich}
\begin{frame}
\textit{
\begin{itemize}
\item Laufzeiten
\end{itemize}
}
\end{frame}

\begin{frame}
\bibliography{../../references}
\bibliographystyle{ieeetr}
\end{frame}

\end{document}
