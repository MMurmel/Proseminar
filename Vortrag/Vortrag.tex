\documentclass{beamer}

%DEFAULT
\usepackage[utf8]{inputenc}
\usepackage[german]{babel}
\usepackage[T1]{fontenc}
\usepackage{amsmath}
\usepackage{amsfonts}
\usepackage{amssymb}

%theming
\usetheme{Dresden}
\usecolortheme{dolphin}

%GRAPHING
\usepackage{tikz-network}
%BIBTEX
\usepackage[numbers,square]{natbib}

%META-INFORMATION
\author{Maximilian Moeller}
\title{Maximaler Fluss in Flussnetzwerken}
\date{\today}

%ALIAS
\newcommand{\ff}{\textsc{Ford-Fulkerson}}
\newcommand{\pr}{Push/Relabel}

\begin{document}
\begin{frame}
\maketitle
\center\large Proseminar Theoretische Informatik 2020
\end{frame}

\begin{frame}
\frametitle{Gliederung}
\tableofcontents
\end{frame}


\section{Motivation}
\begin{frame}
\textit{
\begin{itemize}
\item Grafik: Graph mit Kapazitäten
\item Einsatz überall dort, wo Mengen auf mehreren Wegen transportiert werden
\item Beispiel Rechnernetze: wie viele Pakete (= Informationen) können von A nach B fließen
\item Unterscheidung vom Kapazitätsproblem
\item Verweis auf \citep{Cormen09}
\end{itemize}
}
\end{frame}


\section{Grundlagen}
\subsection{Flussnetzwerke}
\begin{frame}
\frametitle{Flussnetzwerke}
\textit{
\begin{itemize}
\item Grafik: Flussnetzwerk (FN)
\item verbotene Spezialfälle (anti-parallel, reflexiv)
\item Quelle, Senke, $s \to v \to t$
\end{itemize}
}
\end{frame}

\begin{frame}
\frametitle{Spezialfälle}
\textit{
\begin{itemize}
\item FN mit antiparallelen Kanten
\item FN mit mehreren Quellen/Senken
\end{itemize}
}
\end{frame}


\subsection{Fluss}
\begin{frame}
\frametitle{Fluss in Flussnetzwerken}
\textit{
\begin{itemize}
\item Grafik: Flussnetzwerk mit Fluss
\item Notation einführen
\item Anforderungen an einen Fluss
\end{itemize}
}
\end{frame}

\begin{frame}
\frametitle{Maximaler-Fluss-Problem}
\textit{
\begin{itemize}
\item Wert eines Flusses
\item Grafiken mit leicht erkennbaren Werten der Flüsse
\item maximaler-Fluss-Problem
\end{itemize}
}
\end{frame}


\section{weitere Theorie}
\subsection{Restnetzwerke}
\begin{frame}
\frametitle{Restnetzwerke}
\textit{
\begin{itemize}
\item Grafik: Restnetzwerk aus Flussnetzwerk und Fluss
\item Erklärung: RN sind keine FN, aber Fluss trotzdem analog
\item Erhöhung $f\uparrow f'$
\end{itemize}
}
\end{frame}

\subsection{Erweiterungspfade}
\begin{frame}
\frametitle{Erweiterungspfade}
\textit{
\begin{itemize}
\item Grafik: Erweiterungspfad in $G_{f}$
\item Erklärung: $f_{p}$ und $\lvert f\uparrow f_{p} \rvert > \lvert f\rvert$
\item Überleitung zu \ff
\end{itemize}
}
\end{frame}


\section{Algorithmen}
\subsection{Ford-Fulkerson}
\begin{frame}
\frametitle{\ff}
\textit{
\begin{itemize}
\item Erklärung Algorithmus mit Vorwissen aus RN und Erweiterungspfade
\item maximal? -> ja, wegen maxflow-mincut-Theorem (Erklärung reduzieren auf $\lvert f\rvert$ max. $\Leftrightarrow$ kein Pfad in $G_{f}$)
\item Terminierung
\item Verbesserungen (Edmonds-Karp-Algorithmus)
\end{itemize}
}
\end{frame}

\subsection{Push-Relabel}
\begin{frame}
\frametitle{\pr}
\textit{
\begin{itemize}
\item Intuition
\item Vorfluss -> verletzt ggf. Flusserhaltung
\item Höhe
\item Operationen
\item generischer \pr -Algorithmus
\item Verbesserungen (Relabel-to-Front)
\end{itemize}
}
\end{frame}

\subsection{Vergleich}
\begin{frame}
\textit{
\begin{itemize}
\item Laufzeiten
\end{itemize}
}
\end{frame}

\begin{frame}
\bibliography{../../references}
\bibliographystyle{ieeetr}
\end{frame}

\end{document}
