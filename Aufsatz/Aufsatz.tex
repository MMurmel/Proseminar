\documentclass[12pt,a4paper,titlepage,onecolumn,ngerman,draft]{scrartcl}

%META-INFORMATION
\author{Maximilian Moeller}
\title{Maximaler Fluss in Flussnetzwerken}
\subtitle{Proseminar Theoretische Informatik 2020}
\date{\today}

%DEFAULT
\usepackage[utf8]{inputenc}
\usepackage[german]{babel}
\usepackage[T1]{fontenc}
\usepackage{mathtools}
\usepackage{amsmath}
\usepackage{amsfonts}
\usepackage{amssymb}

%GRAPHING
\usepackage{tikz-network}
\usepackage{adjustbox}
\usepackage{subcaption}

%BIBTEX
\usepackage[numbers,square]{natbib}
\usepackage{url}
%smaller URLs
\renewcommand{\UrlFont}{\small\tt}

%ALIAS
\newcommand{\ff}{Ford-Fulkerson}
\newcommand{\pr}{Push/Relabel}

\begin{document}
\maketitle
\nocite{Cormen09}

\tableofcontents

\begin{abstract}
\end{abstract}

\section{Grundlagen}
In diesem Abschnitt werden die grundlegenen Voraussetzungen geschaffen, um das Problem des maximalen Flusses formulieren zu können.

\subsection{Flussnetzwerk}
Ein Flussnetzwerk ist ein gerichteter Graph $G = (V,E)$, in dem jeder Kante $(u,v) \in E$ durch eine Kapazitätsfunktion $c: V\times V\to\mathfrak{B}$ eine Kapazität $c(u,v) \geq 0$ zugeordnet ist.
Dabei ist häufig $\mathfrak{B} = \mathbb{N}$, jedoch ist auch $\mathfrak{B} = \mathbb{Z}$ oder $\mathfrak{B} = \mathbb{R}$ möglich.
Für $(u,v) \notin E$ sei $c(u,v) = 0$.
Darüber hinaus fordern wir: 
\begin{equation} \label{noparallel}
\forall (u,v)\in V\times V: (u,v)\in E \Rightarrow (v,u)\notin E\\
\end{equation}


\newpage
\section{\ff}
Zunächst soll die Klasse der \ff -Algorithmen vorgestellt werden.

\newpage
\bibliography{../../references}{}
\bibliographystyle{ieeetr}
\end{document}
