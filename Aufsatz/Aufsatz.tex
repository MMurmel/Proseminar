\documentclass[12pt,a4paper,titlepage,onecolumn,ngerman,draft]{scrartcl}

%META-INFORMATION
\author{Maximilian Moeller}
\title{Maximaler Fluss in Flussnetzwerken}
\subtitle{Proseminar Theoretische Informatik 2020}
\date{\today}

%PACKAGES
\usepackage[utf8]{inputenc}
\usepackage[german]{babel}
\usepackage[T1]{fontenc}
\usepackage{mathtools}
\usepackage{amsmath}
\usepackage{amsfonts}
\usepackage{amssymb}
\usepackage{algorithm}
\usepackage[noend]{algpseudocode}

%OPTIONS
%no "then" or "do" in algorithm
\renewcommand\algorithmicthen{}
\renewcommand\algorithmicdo{}

%GRAPHING
\usepackage{tikz-network}
\usepackage{caption}
\usepackage{subcaption}

%BIBTEX
%TODO check for better bibliography package
\usepackage[numbers,square]{natbib}
\usepackage{url}
%smaller URLs
\renewcommand{\UrlFont}{\small\tt}

%ALIAS
\newcommand{\ff}{Ford-Fulkerson}
\newcommand{\pr}{Push/Relabel}

\begin{document}
\maketitle
\nocite{Cormen09}

\tableofcontents

\begin{abstract}
In dieser Ausarbeitung soll eine Zusammenfassung des Kapitels 26 \glqq Maximaler Fluss\grqq{} aus dem Buch \glqq Introduction to Algorithms \grqq{} \citep{Cormen09} gegeben werden. 
Dabei soll insbesondere auf den \pr -Algorithmus eingegangen und ein Vergleich zum \ff -Algorithmus gezogen werden.

Die Fragestellung nach dem maximalen Fluss ist ein Problem aus der Graphentheorie, das
in vielen praktischen Fällen Anwendung findet. 
Überall dort, wo die Menge einer zwischen verschiedenen Punkten transportierten Ware maximiert werden soll, kann auf die hier entwickelten Lösungen zurückgegriffen werden. 
Als Beispiele seien die Informationsübertragung in Computernetzwerken oder der Flüssigkeitstransport in Rohrleitungssystemen genannt. 
Besonders an letzterem lassen sich die Intuitionen hinter den Formeln gut erklären. Solche Erklärungen werden hier durch \textit{kursiven Text} angezeigt.
\end{abstract}

\section{Grundlagen}
In diesem Abschnitt werden die grundlegenden Voraussetzungen geschaffen, um das Problem des maximalen Flusses sinnvoll formulieren zu können.

\subsection{Flussnetzwerke}
Ein Flussnetzwerk ist ein gerichteter Graph $G = (V,E)$, in dem jeder Kante $(u,v) \in E$ durch eine Kapazitätsfunktion $c: V\times V\to\mathfrak{B}$ eine Kapazität $c(u,v) \geq 0$ zugeordnet ist.
Dabei ist häufig $\mathfrak{B} = \mathbb{N}$, seltener auch $\mathfrak{B} = \mathbb{Z}$ oder $\mathfrak{B} = \mathbb{R}$.
Für $(u,v) \notin E$ sei $c(u,v) = 0$.
Außerdem fordern wir:
\begin{equation} \label{no_parallel}
\forall (u,v)\in V\times V: (u,v)\in E \Rightarrow (v,u)\notin E\\
\end{equation}
Wir verbieten also entgegengerichtete Kantenpaare.
Darüber hinaus darf es in einem Flussnetzwerk auch keine reflexiven Kanten -- also Kanten mit gleichem Start- und Zielknoten -- geben.
Statt zu zeigen, dass dies keine echten Einschränkungen sind, sei hier auf \cite[Kapitel, 26]{Cormen09} verwiesen.

Jedes Flussnetzwerk enthält zwei besondere Knoten: Eine Quelle $s$ und eine Senke $t$.
Der Einfachheit halber liege jeder Knoten $v \in V$ auf einem Pfad von $s$ nach $t$.
Grafik 1.a) zeigt ein Beispiel für ein Flussnetzwerk.

\subsection{Fluss in Flussnetzwerken}
Ein Fluss $f$ in einem Flussnetzwerk $G=(V,E)$ ist eine Funktion $f: V \times V \to \mathfrak{B}$. Von nun an soll $\mathfrak{B} = \mathbb{N}$ betrachtet werden.
Dabei muss ein Fluss drei besondere Bedingungen erfüllen:
\begin{align}
&\forall u,v\in V\colon (u,v)\notin E\Rightarrow f(u,v) = 0 \label{only_on_edges} \\
&\forall u,v\in V\colon 0\leq f(u,v)\leq c(u,v) \label{limited_by_capacity} \\
&\forall u\in V\setminus\{s,t\}\colon\sum_{v\in V} f(v,u) = \sum_{v\in V} f(u,v) \label{flow_continuation}
\end{align}
Zur Erklärung dieser Formeln: \\
\textit{
\eqref{only_on_edges} besagt, dass Wasser nur dort fließen kann, wo auch Rohre liegen. \\
\eqref{limited_by_capacity} besagt, dass nicht mehr Wasser durch ein Rohr fließen kann, als dessen Kapazität hergibt. \\
\eqref{flow_continuation} besagt, dass aus jeder Station (außer Quelle und Senke) genau so viel Wasser herausfließt, wie hinein. Insbesondere kann sich also in keiner Station zwischen $s$ und $t$ Wasser ansammeln. Diese Eigenschaft heißt \textbf{Flusserhaltung} und entspricht im Beispiel des Stromflusses der Kirchhoffschen Knotenregel.}

\begin{figure}
\centering
\begin{subfigure}[b]{0.49\textwidth}
    \begin{tikzpicture}
    \Vertex[label=$s$]{s}
    \Vertex[label=$v_1$, x=2, y=1]{1}
    \Vertex[label=$v_2$, x=2, y=-1]{2}
    \Vertex[label=$v_3$, x=4, y=1]{3}
    \Vertex[label=$v_4$, x=4, y=-1]{4}
    \Vertex[label=$t$, x=6, y=0]{t}
    \Edge[Direct, label=4](s)(1)
    \Edge[Direct, label=7](s)(2)
    \Edge[Direct, label=2](1)(3)
    \Edge[Direct, label=4](2)(3)
    \Edge[Direct, label=3](2)(4)
    \Edge[Direct, label=6](3)(t)
    \Edge[Direct, label=2](4)(t)
    \end{tikzpicture}
    %TODO
    \caption{a}
    \label{fig:1a}
\end{subfigure}
\begin{subfigure}[b]{0.49\textwidth}
    \begin{tikzpicture}
        \Vertex[label=$s$]{s}
        \Vertex[label=$v_1$, x=2, y=1]{1}
        \Vertex[label=$v_2$, x=2, y=-1]{2}
        \Vertex[label=$v_3$, x=4, y=1]{3}
        \Vertex[label=$v_4$, x=4, y=-1]{4}
        \Vertex[label=$t$, x=6, y=0]{t}
        \Edge[Direct, label=4](s)(1)
        \Edge[Direct, label=7](s)(2)
        \Edge[Direct, label=2](1)(3)
        \Edge[Direct, label=4](2)(3)
        \Edge[Direct, label=3](2)(4)
        \Edge[Direct, label=6](3)(t)
        \Edge[Direct, label=2](4)(t)
        \end{tikzpicture}
            %TODO
        \caption{b}
        \label{fig:1b}
\end{subfigure}
\caption{Abbildung 1}
\label{fig:1}
\end{figure}

\section{Weiterführende Theorie}
\subsection{Restnetzwerke}
\subsection{Erweiterungspfade}

\section{\ff -Algorithmen}
Zunächst soll die Klasse der \ff -Algorithmen vorgestellt werden.

\section{\pr -Algorithmen}

\newpage
\bibliography{../../references}{}
\bibliographystyle{ieeetr}
\end{document}
