\documentclass[12pt,a4paper,titlepage,onecolumn,ngerman,draft]{scrartcl}

%META-INFORMATION
\author{Maximilian Moeller}
\title{Maximaler Fluss in Flussnetzwerken}
\subtitle{Proseminar Theoretische Informatik 2020}
\date{\today}

%DEFAULT
\usepackage[utf8]{inputenc}
\usepackage[german]{babel}
\usepackage[T1]{fontenc}
\usepackage{mathtools}
\usepackage{amsmath}
\usepackage{amsfonts}
\usepackage{amssymb}

%GRAPHING
\usepackage{tikz-network}
\usepackage{adjustbox}
\usepackage{subcaption}

%BIBTEX
\usepackage[numbers,square]{natbib}
\usepackage{url}
%smaller URLs
\renewcommand{\UrlFont}{\small\tt}

%ALIAS
\newcommand{\ff}{Ford-Fulkerson}
\newcommand{\pr}{Push/Relabel}

\begin{document}
\maketitle
\nocite{Cormen09}

\tableofcontents

\begin{abstract}
In dieser Ausarbeitung soll eine Zusammenfassung des Kapitels 26 \glqq Maximaler Fluss\grqq{} aus dem Buch \glqq Introduction to Algorithms \grqq{} \citep{Cormen09} gegeben werden. 
Dabei soll insbesondere auf den \pr -Algorithmus eingegangen und ein Vergleich zum \ff -Algorithmus gezogen werden.
\end{abstract}

\section{Grundlagen}
In diesem Abschnitt werden die grundlegenden Voraussetzungen geschaffen, um das Problem des maximalen Flusses sinnvoll formulieren zu können.

\subsection{Flussnetzwerke}
Ein Flussnetzwerk ist ein gerichteter Graph $G = (V,E)$, in dem jeder Kante $(u,v) \in E$ durch eine Kapazitätsfunktion $c: V\times V\to\mathfrak{B}$ eine Kapazität $c(u,v) \geq 0$ zugeordnet ist.
Dabei ist häufig $\mathfrak{B} = \mathbb{N}$, seltener auch $\mathfrak{B} = \mathbb{Z}$ oder $\mathfrak{B} = \mathbb{R}$.
Für $(u,v) \notin E$ sei $c(u,v) = 0$.
Außerdem fordern wir:
\begin{equation} \label{no_parallel}
\forall (u,v)\in V\times V: (u,v)\in E \Rightarrow (v,u)\notin E\\
\end{equation}
Wir verbieten also entgegengerichtete Kantenpaare.
Darüber hinaus darf es in einem Flussnetzwerk auch keine reflexiven Kanten -- also Kanten mit gleichem Start- und Zielknoten -- geben.

Jedes Flussnetzwerk enthält zwei besondere Knoten: Eine Quelle $s$ und eine Senke $t$.
Der Einfachheit halber liege jeder Knoten $v \in V$ auf einem Pfad von $s$ nach $t$.

Grafik 1.a) zeigt ein Beispiel für ein Flussnetzwerk.

\subsection{Fluss in Flussnetzwerken}
Ein Fluss $f$ in einem Flussnetzwerk $G=(V,E)$ ist eine Funktion $f: V \times V \to \mathfrak{B}$. Von nun an soll $\mathfrak{B} = \mathbb{N}$ betrachtet werden.
Dabei muss ein Fluss drei besondere Bedingungen erfüllen:
\begin{align}
&\forall u,v\in V\colon (u,v)\notin E\Rightarrow f(u,v) = 0 \label{only_on_edges} \\
&\forall u,v\in V\colon 0\leq f(u,v)\leq c(u,v) \label{limited_by_capacity} \\
&\forall u\in V\setminus\{s,t\}\colon\sum_{v\in V} f(v,u) = \sum_{v\in V} f(u,v) \label{flow_continuation}
\end{align}

\section{Weiterführende Theorie}
\subsection{Restnetzwerke}
\subsection{Erweiterungspfade}

\section{\ff -Algorithmen}
Zunächst soll die Klasse der \ff -Algorithmen vorgestellt werden.

\section{\pr -Algorithmen}

\newpage
\bibliography{../../references}{}
\bibliographystyle{ieeetr}
\end{document}
