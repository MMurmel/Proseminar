\documentclass[12pt,a4paper,titlepage,onecolumn,ngerman]{scrartcl}

%META-INFORMATION
\author{Maximilian Moeller}
\title{Maximaler Fluss in Flussnetzwerken}
\subtitle{Proseminar Theoretische Informatik 2020}
\date{\today}

%DEFAULT
\usepackage[utf8]{inputenc}
\usepackage[german]{babel}
\usepackage[T1]{fontenc}
\usepackage{mathtools}
\usepackage{amsmath}
\usepackage{amsfonts}
\usepackage{amssymb}
\usepackage{microtype}

%GRAPHING
\usepackage{tikz-network}
\usepackage{adjustbox}
\usepackage{subcaption}

%BIBTEX
\usepackage[numbers,square]{natbib}
\usepackage{url}
%smaller URLs
\renewcommand{\UrlFont}{\small\tt}

%ALIAS
\newcommand{\ff}{Ford-Fulkerson}
\newcommand{\pr}{Push/Relabel}

\begin{document}
\maketitle
\nocite{Cormen09}

\tableofcontents

\begin{abstract}
In dieser Ausarbeitung soll eine Zusammenfassung des Kapitels 26 \glqq Maximaler Fluss\grqq{} aus dem Buch \glqq Introduction to Algorithms \grqq{} \citep{Cormen09} gegeben werden. 
Dabei soll insbesondere auf den \pr -Algorithmus eingegangen und ein Vergleich zum \ff -Algorithmus gezogen werden.
\end{abstract}

\section{Grundlagen}
In diesem Abschnitt werden die grundlegenden Voraussetzungen geschaffen, um das Problem des maximalen Flusses sinnvoll formulieren zu können.

\subsection{Flussnetzwerke}
Ein Flussnetzwerk ist ein gerichteter Graph $G = (V,E)$, in dem jeder Kante $(u,v) \in E$ durch eine Kapazitätsfunktion $c: V\times V\to\mathfrak{B}$ eine Kapazität $c(u,v) \geq 0$ zugeordnet ist.
Dabei ist häufig $\mathfrak{B} = \mathbb{N}$, seltener auch $\mathfrak{B} = \mathbb{Z}$ oder $\mathfrak{B} = \mathbb{R}$.
Im Folgenden soll deswegen $\mathfrak{B} = \mathbb{N}$ betrachtet werden.
Für $(u,v) \notin E$ sei $c(u,v) = 0$.
Außerdem fordern wir:
\begin{equation} \label{no_parallel}
\forall (u,v)\in V\times V: (u,v)\in E \Rightarrow (v,u)\notin E\\
\end{equation}
Wir verbieten also entgegengerichtete Kantenpaare.
Darüber hinaus darf es in einem Flussnetzwerk auch keine reflexiven Kanten -- also Kanten mit gleichem Start- und Zielknoten -- geben.

Jedes Flussnetzwerk enthält zwei besondere Knoten: Eine Quelle $s$ und eine Senke $t$.
Der Einfachheit halber liege jeder Knoten $v \in V$ auf einem Pfad von $s$ nach $t$.

Abbildung \ref{fig:1.1} zeigt ein Beispiel für ein Flussnetzwerk.

\subsection{Fluss in Flussnetzwerken}
Ein Fluss $f$ in einem Flussnetzwerk $G=(V,E)$ ist eine Funktion $f: V \times V \to \mathfrak{B}$.
Dabei muss ein Fluss drei besondere Bedingungen erfüllen:
\begin{align}
&\forall u,v\in V\colon (u,v)\notin E\Rightarrow f(u,v) = 0 \label{only_on_edges} \\
&\forall u,v\in V\colon 0\leq f(u,v)\leq c(u,v) \label{limited_by_capacity} \\
&\forall u\in V\setminus\{s,t\}\colon\sum_{v\in V} f(v,u) = \sum_{v\in V} f(u,v) \label{flow_continuation}
\end{align}
Die Eigenschaft \eqref{flow_continuation} trägt den Namen Flusserhaltung. 
Im Beispiel des Stromflusses entspricht sie der Kirchhoffschen Knotenregel.

Abbildung \ref{fig:1.2} zeigt einen Fluss.

\subsection{Wert eines Flusses}
Der Wert $\lvert f \rvert$ eines Flusses $f$ ist definiert als:
\begin{equation}\label{flow_value}
\lvert f \rvert := \sum_{u\in V} f(s,u) - \sum_{u\in V} f(u,s)
\end{equation}
Der Wert eines Flusses ist also die Menge an Fluss, die \glqq netto\grqq{} aus der Quelle $s$ herausfließt.
Wegen \eqref{flow_continuation} ist gleich der Menge an Fluss, die (\glqq netto\grqq{}) in die Senke $t$ hineinfließt.
Häufig enthält ein Flussnetzwerk keine eingehenden Kanten zur Quelle, womit sich die Gleichung zu $\lvert f \rvert := \sum_{u\in V} f(s,u)$ vereinfacht.

Im maximaler-Fluss-Problem ist ein Flussnetzwerk $G$ mit Quelle $s$ und Senke $t$ gegeben.
Gesucht wird ein Fluss $f$, dessen Wert $\lvert f\rvert$ maximal für dieses Flussnetzwerk ist.
Dieser Fluss muss nicht eindeutig bestimmt sein. %TODO Bild

\begin{figure}\label{fig:1}
    \centering
    \begin{subfigure}[t]{.475\textwidth}
        \centering
        \begin{tikzpicture}
            \Vertex[label=$s$]{s}
            \Vertex[label=$u_1$, x=2, y=1]{1}
            \Vertex[label=$u_2$, x=2, y=-1]{2}
            \Vertex[label=$u_3$, x=4, y=1]{3}
            \Vertex[label=$u_4$, x=4, y=-1]{4}
            \Vertex[label=$t$, x=6, y=0]{t}
            \Edge[Direct, label=4](s)(1)
            \Edge[Direct, label=7](s)(2)
            \Edge[Direct, label=2](1)(3)
            \Edge[Direct, label=4](2)(3)
            \Edge[Direct, label=3](2)(4)
            \Edge[Direct, label=6](3)(t)
            \Edge[Direct, label=2](4)(t)
        \end{tikzpicture}
        \caption{Ein Flussnetzwerk $G$}
        \label{fig:1.1}
    \end{subfigure}
    \hfill
    \begin{subfigure}[t]{.475\textwidth}
        \centering
        \begin{tikzpicture}
            \Vertex[label=$s$]{s}
            \Vertex[label=$u_1$, x=2, y=1]{1}
            \Vertex[label=$u_2$, x=2, y=-1]{2}
            \Vertex[label=$u_3$, x=4, y=1]{3}
            \Vertex[label=$u_4$, x=4, y=-1]{4}
            \Vertex[label=$t$, x=6, y=0]{t}
            \Edge[Direct, label=1/4](s)(1)
            \Edge[Direct, label=2/7](s)(2)
            \Edge[Direct, label=1/2](1)(3)
            \Edge[Direct, label=4](2)(3)
            \Edge[Direct, label=2/3](2)(4)
            \Edge[Direct, label=1/6](3)(t)
            \Edge[Direct, label=2/2](4)(t)
        \end{tikzpicture}
        \caption{Ein Fluss $f$ in $G$ mit $\lvert f\rvert = 3$}
        \label{fig:1.2}
    \end{subfigure}
    \caption{In der hier verwendeten Darstellung werden Kanten $(u,v)$ mit $f(u,v)/c(u,v)$ beschrifet. Ist $f(u,v) = 0$, so wird nur die Kapazität angegeben.}
\end{figure}

\section{Weiterführende Theorie}
\subsection{Restnetzwerke}
Restnetzwerke sind ein Konstrukt, das es uns erlauben wird effizient Kanten zu finden, auf denen der Fluss noch erhöht werden kann.
Restnetzwerke sind ebenfalls gerichtete Graphen. Sie existieren jedoch nicht für sich, sondern werden durch den Fluss in einem Restnetzwerk induziert.

Sei $G=(V,E)$ ein Flussnetzwerk und $f$ ein Fluss in $G$.
Die Restkapazität einer Kante ist definiert als:
\begin{equation}\label{residual_capacity}
    c_{f}(u,v) = 
    \begin{cases}
        c(u,v) - f(u,v) & \text{falls $(u,v)\in E$}\\
        f(v,u) & \text{falls $(v,u) \in E$}\\
        0 & \text{sonst}
    \end{cases}
\end{equation}
Wegen \eqref{no_parallel} tritt immer nur genau einer dieser Fälle ein.
Daraus können wir die Menge der Restkanten definieren:
\begin{equation}\label{residual_edges}
    E_f = \{ (u,v) \in V \times V \mid c_{f}(u,v) > 0\}
\end{equation}
Dann ist das durch $f$ induzierte Restnetzwerk $G_f=(V,E_f)$

Es ist wichtig zu verstehen, dass Restnetzwerke keine Flussnetzwerke sind.
\section{\ff -Algorithmen}
Zunächst soll die Klasse der \ff -Algorithmen vorgestellt werden.

\section{\pr -Algorithmen}

\newpage
\bibliography{../../references}{}
\bibliographystyle{ieeetr}
\end{document}
